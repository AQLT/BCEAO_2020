\PassOptionsToPackage{unicode=true}{hyperref} % options for packages loaded elsewhere
\PassOptionsToPackage{hyphens}{url}
%
\documentclass[10pt,xcolor=table,color={dvipsnames,usenames},ignorenonframetext,usepdftitle=false,french]{beamer}
\setbeamertemplate{caption}[numbered]
\setbeamertemplate{caption label separator}{: }
\setbeamercolor{caption name}{fg=normal text.fg}
\beamertemplatenavigationsymbolsempty
\usepackage{caption}
\captionsetup{skip=0pt,belowskip=0pt}
%\setlength\abovecaptionskip{-15pt}
\usepackage{lmodern}
\usepackage{amssymb,amsmath,mathtools,multirow}
\usepackage{float,hhline}
\usepackage{tikz,pgfplots}
\usepackage[tikz]{bclogo}
\usepackage{ifxetex,ifluatex}
\usepackage{fixltx2e} % provides \textsubscript
\ifnum 0\ifxetex 1\fi\ifluatex 1\fi=0 % if pdftex
  \usepackage[T1]{fontenc}
  \usepackage[utf8]{inputenc}
  \usepackage{textcomp} % provides euro and other symbols
\else % if luatex or xelatex
  \usepackage{unicode-math}
  \defaultfontfeatures{Ligatures=TeX,Scale=MatchLowercase}
\fi
\usetheme[coding=utf8,language=french,
,titlepagelogo=img/logoinsee
,secondlogo=true
]{TorinoTh}
\titlepagesecondlogo{img/BCEAO}
% use upquote if available, for straight quotes in verbatim environments
\IfFileExists{upquote.sty}{\usepackage{upquote}}{}
% use microtype if available
\IfFileExists{microtype.sty}{%
\usepackage[]{microtype}
\UseMicrotypeSet[protrusion]{basicmath} % disable protrusion for tt fonts
}{}
\IfFileExists{parskip.sty}{%
\usepackage{parskip}
}{% else
\setlength{\parindent}{0pt}
\setlength{\parskip}{6pt plus 2pt minus 1pt}
}
\usepackage{hyperref}
\hypersetup{
            pdftitle={0 - Introduction},
            pdfauthor={Dominique Ladiray et Alain Quartier-la-Tente},
            pdfborder={0 0 0},
            breaklinks=true}
\urlstyle{same}  % don't use monospace font for urls
\newif\ifbibliography
% Prevent slide breaks in the middle of a paragraph:
\widowpenalties 1 10000
\raggedbottom
\AtBeginPart{
  \let\insertpartnumber\relax
  \let\partname\relax
  \frame{\partpage}
}
\AtBeginSection{
  \ifbibliography
  \else
    \begin{frame}{Sommaire}
    \tableofcontents[currentsection, hideothersubsections]
    \end{frame}
  \fi
}
\setlength{\emergencystretch}{3em}  % prevent overfull lines
\providecommand{\tightlist}{%
  %\setlength{\itemsep}{0pt}
  \setlength{\parskip}{0pt}
  }
\setcounter{secnumdepth}{0}

% set default figure placement to htbp
\makeatletter
\def\fps@figure{htbp}
\makeatother

\usepackage{animate}
\usepackage{fontawesome5}

\title{0 - Introduction}
\ateneo{BCEAO - 20 au 25 janvier 2019}
\author{Dominique Ladiray et Alain Quartier-la-Tente}
\date{}


\setrellabel{}

\setcandidatelabel{}

\rel{}
\division{(\href{mailto:dominique.ladiray@insee.fr}{\nolinkurl{dominique.ladiray@insee.fr}}
et
\href{mailto:alain.quartier@yahoo.fr}{\nolinkurl{alain.quartier@yahoo.fr}})}

\departement{}

\DeclareMathOperator{\Cov}{Cov}
\newcommand{\E}[1]{\mathbb{E}\left[ #1 \right]}
\newcommand{\V}[1]{\mathbb{V}\left[ #1 \right]}
\newcommand{\cov}[2]{\Cov\left( #1\,,\,#2 \right)}

\begin{document}
\frame[plain,noframenumbering]{\titlepage}

\begin{frame}{Introduction et programme}
\protect\hypertarget{introduction-et-programme}{}

Objectif : adapter au mieux la formation à vos besoins !

\(\rightarrow\) N'hésitez pas à échanger avec nous, entre vous, à
exprimer des besoins, demander des clarifications\ldots{}

\pause

Programme indicatif et révisable :

\begin{itemize}
\item
  Lundi 20 janvier :

  \begin{itemize}
  \item
    Installation des logiciels/packages
  \item
    TP introductif à la manipulation des séries temporelles
  \item
    Introduction à la désaisonnalisation sous R
  \end{itemize}

  \pause
\item
  Mardi 21 janvier :

  \begin{itemize}
  \item
    Plug-in \emph{nowcasting} de la banque nationale de belgique
  \item
    Modèles SARIMA et ECM sous R
  \end{itemize}
\end{itemize}

\end{frame}

\begin{frame}{Introduction et programme}
\protect\hypertarget{introduction-et-programme-1}{}

\begin{itemize}
\item
  Mercredi 22 janvier :

  \begin{itemize}
  \item
    Construction d'indicateurs de retournement conjoncturels
  \item
    Partie libre à définir
  \end{itemize}
\item
  Jeudi 23 et vendredi 24 janvier (Dominique Ladiray) :

  \begin{itemize}
  \tightlist
  \item
    Désaisonnalisation
  \end{itemize}
\end{itemize}

\medskip

Ensemble des documents disponibles sous :
\url{https://github.com/AQLT/BCEAO_2020}

\end{frame}

\begin{frame}{Deux travaux pratiques}
\protect\hypertarget{deux-travaux-pratiques}{}

\begin{itemize}
\item
  TP0 : traitement des objets séries temporelles sous R

  \begin{itemize}
  \item
    Créer des séries temporelles, les raccourcir, manipuler les périodes
  \item
    Faire des manipulations plus ``poussées'' : trimestrialisation,
    imputation des données manquantes, formatage des dates
  \end{itemize}
\item
  TP1 : Désaisonnalisation sous R avec JDemetra+

  \begin{itemize}
  \item
    Rafraichir son workspace et exporter les résultats
  \item
    Désaisonnaliser sous R, créer et importer les workspaces de
    JDemetra+
  \end{itemize}

  \pause

  \begin{itemize}
  \tightlist
  \item
    Récupérer des stickers
  \end{itemize}
\end{itemize}

\end{frame}

\end{document}
